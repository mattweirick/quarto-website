% Options for packages loaded elsewhere
\PassOptionsToPackage{unicode}{hyperref}
\PassOptionsToPackage{hyphens}{url}
\PassOptionsToPackage{dvipsnames,svgnames,x11names}{xcolor}
%
\documentclass[
  twocolumn]{article}

\usepackage{amsmath,amssymb}
\usepackage{iftex}
\ifPDFTeX
  \usepackage[T1]{fontenc}
  \usepackage[utf8]{inputenc}
  \usepackage{textcomp} % provide euro and other symbols
\else % if luatex or xetex
  \usepackage{unicode-math}
  \defaultfontfeatures{Scale=MatchLowercase}
  \defaultfontfeatures[\rmfamily]{Ligatures=TeX,Scale=1}
\fi
\usepackage{lmodern}
\ifPDFTeX\else  
    % xetex/luatex font selection
  \setmainfont[]{Helvetica}
  \setsansfont[]{Helvetica}
  \setmonofont[]{Helvetica}
\fi
% Use upquote if available, for straight quotes in verbatim environments
\IfFileExists{upquote.sty}{\usepackage{upquote}}{}
\IfFileExists{microtype.sty}{% use microtype if available
  \usepackage[]{microtype}
  \UseMicrotypeSet[protrusion]{basicmath} % disable protrusion for tt fonts
}{}
\makeatletter
\@ifundefined{KOMAClassName}{% if non-KOMA class
  \IfFileExists{parskip.sty}{%
    \usepackage{parskip}
  }{% else
    \setlength{\parindent}{0pt}
    \setlength{\parskip}{6pt plus 2pt minus 1pt}}
}{% if KOMA class
  \KOMAoptions{parskip=half}}
\makeatother
\usepackage{xcolor}
\setlength{\emergencystretch}{3em} % prevent overfull lines
\setcounter{secnumdepth}{-\maxdimen} % remove section numbering
% Make \paragraph and \subparagraph free-standing
\ifx\paragraph\undefined\else
  \let\oldparagraph\paragraph
  \renewcommand{\paragraph}[1]{\oldparagraph{#1}\mbox{}}
\fi
\ifx\subparagraph\undefined\else
  \let\oldsubparagraph\subparagraph
  \renewcommand{\subparagraph}[1]{\oldsubparagraph{#1}\mbox{}}
\fi


\providecommand{\tightlist}{%
  \setlength{\itemsep}{0pt}\setlength{\parskip}{0pt}}\usepackage{longtable,booktabs,array}
\usepackage{calc} % for calculating minipage widths
% Correct order of tables after \paragraph or \subparagraph
\usepackage{etoolbox}
\makeatletter
\patchcmd\longtable{\par}{\if@noskipsec\mbox{}\fi\par}{}{}
\makeatother
% Allow footnotes in longtable head/foot
\IfFileExists{footnotehyper.sty}{\usepackage{footnotehyper}}{\usepackage{footnote}}
\makesavenoteenv{longtable}
\usepackage{graphicx}
\makeatletter
\def\maxwidth{\ifdim\Gin@nat@width>\linewidth\linewidth\else\Gin@nat@width\fi}
\def\maxheight{\ifdim\Gin@nat@height>\textheight\textheight\else\Gin@nat@height\fi}
\makeatother
% Scale images if necessary, so that they will not overflow the page
% margins by default, and it is still possible to overwrite the defaults
% using explicit options in \includegraphics[width, height, ...]{}
\setkeys{Gin}{width=\maxwidth,height=\maxheight,keepaspectratio}
% Set default figure placement to htbp
\makeatletter
\def\fps@figure{htbp}
\makeatother
% definitions for citeproc citations
\NewDocumentCommand\citeproctext{}{}
\NewDocumentCommand\citeproc{mm}{%
  \begingroup\def\citeproctext{#2}\cite{#1}\endgroup}
\makeatletter
 % allow citations to break across lines
 \let\@cite@ofmt\@firstofone
 % avoid brackets around text for \cite:
 \def\@biblabel#1{}
 \def\@cite#1#2{{#1\if@tempswa , #2\fi}}
\makeatother
\newlength{\cslhangindent}
\setlength{\cslhangindent}{1.5em}
\newlength{\csllabelwidth}
\setlength{\csllabelwidth}{3em}
\newenvironment{CSLReferences}[2] % #1 hanging-indent, #2 entry-spacing
 {\begin{list}{}{%
  \setlength{\itemindent}{0pt}
  \setlength{\leftmargin}{0pt}
  \setlength{\parsep}{0pt}
  % turn on hanging indent if param 1 is 1
  \ifodd #1
   \setlength{\leftmargin}{\cslhangindent}
   \setlength{\itemindent}{-1\cslhangindent}
  \fi
  % set entry spacing
  \setlength{\itemsep}{#2\baselineskip}}}
 {\end{list}}
\usepackage{calc}
\newcommand{\CSLBlock}[1]{\hfill\break\parbox[t]{\linewidth}{\strut\ignorespaces#1\strut}}
\newcommand{\CSLLeftMargin}[1]{\parbox[t]{\csllabelwidth}{\strut#1\strut}}
\newcommand{\CSLRightInline}[1]{\parbox[t]{\linewidth - \csllabelwidth}{\strut#1\strut}}
\newcommand{\CSLIndent}[1]{\hspace{\cslhangindent}#1}

\usepackage{float}
\makeatletter
\let\oldlt\longtable
\let\endoldlt\endlongtable
\def\longtable{\@ifnextchar[\longtable@i \longtable@ii}
\def\longtable@i[#1]{\begin{figure}[H]
\onecolumn
\begin{minipage}{0.5\textwidth}
\oldlt[#1]
}
\def\longtable@ii{\begin{figure}[H]
\onecolumn
\begin{minipage}{0.5\textwidth}
\oldlt
}
\def\endlongtable{\endoldlt
\end{minipage}
\twocolumn
\end{figure}}
\makeatother
\makeatletter
\@ifpackageloaded{caption}{}{\usepackage{caption}}
\AtBeginDocument{%
\ifdefined\contentsname
  \renewcommand*\contentsname{Table of contents}
\else
  \newcommand\contentsname{Table of contents}
\fi
\ifdefined\listfigurename
  \renewcommand*\listfigurename{List of Figures}
\else
  \newcommand\listfigurename{List of Figures}
\fi
\ifdefined\listtablename
  \renewcommand*\listtablename{List of Tables}
\else
  \newcommand\listtablename{List of Tables}
\fi
\ifdefined\figurename
  \renewcommand*\figurename{Figure}
\else
  \newcommand\figurename{Figure}
\fi
\ifdefined\tablename
  \renewcommand*\tablename{Table}
\else
  \newcommand\tablename{Table}
\fi
}
\@ifpackageloaded{float}{}{\usepackage{float}}
\floatstyle{ruled}
\@ifundefined{c@chapter}{\newfloat{codelisting}{h}{lop}}{\newfloat{codelisting}{h}{lop}[chapter]}
\floatname{codelisting}{Listing}
\newcommand*\listoflistings{\listof{codelisting}{List of Listings}}
\makeatother
\makeatletter
\makeatother
\makeatletter
\@ifpackageloaded{caption}{}{\usepackage{caption}}
\@ifpackageloaded{subcaption}{}{\usepackage{subcaption}}
\makeatother
\ifLuaTeX
  \usepackage{selnolig}  % disable illegal ligatures
\fi
\usepackage{bookmark}

\IfFileExists{xurl.sty}{\usepackage{xurl}}{} % add URL line breaks if available
\urlstyle{same} % disable monospaced font for URLs
\hypersetup{
  pdftitle={Job Control and Its Impacts on Burnout in Academic Instruction Librarians},
  colorlinks=true,
  linkcolor={blue},
  filecolor={Maroon},
  citecolor={Blue},
  urlcolor={Blue},
  pdfcreator={LaTeX via pandoc}}

\title{Job Control and Its Impacts on Burnout in Academic Instruction
Librarians}
\author{Matthew Weirick Johnson}
\date{2023-06-12}

\begin{document}
\maketitle
\begin{abstract}
Librarians have been grappling with the issue of burnout for decades, at
least. This study uses the Copenhagen Burnout Inventory (CBI) and Job
Control Inventory to show how job control impacts Burnout. Using the
CBI, Academic Instruction Librarians, on average, have high work-related
burnout and even higher personal burnout compare to other jobs. However,
librarians have low client-related burnout, similar to other caring or
helping professions. The findings point to key factors taht impact job
control and burnout to help consider ways of mitigating burnout and
increasing job control.
\end{abstract}

\subsection{Abstract}\label{abstract}

Librarians have been grappling with the issue of burnout for decades, at
least. This study uses the Copenhagen Burnout Inventory (CBI) and Job
Control Inventory to show how job control impacts burnout. Using the
CBI, academic instruction librarians, on average, have high work-related
burnout and even higher personal burnout compared to other jobs.
However, librarians have low client-related burnout, similar to other
``caring'' or ``helping'' professions. The findings point to key factors
that impact job control and burnout to help consider ways of mitigating
burnout and increasing job control.

\subsection{Introduction}\label{introduction}

Librarians have been grappling with the issue of burnout for decades, at
least, with many acknowledging its prevalence in the profession. In
recent years, additional empirical evidence, both quantitative and
qualitative, has been published. Two specific instances have applied
quantitative inventories to measure burnout among academic librarians.
Applying the Areas of Worklife Survey and Maslach Burnout Inventory,
Nardine (2019) found ``that lack of personal agency is the primary
contributor to a sense of burnout'' (p.~508). Additionally, employing
the Copenhagen Burnout Inventory, Wood et al.~(2020) found that
librarian perceptions of burnout are quite high in comparison to other
occupations, including nurses, hospital doctors, and social workers.
Burnout is clearly a central concern for the profession that has
received considerable attention in scholarship and other discussions,
and agency or job control may be a large contributor.

Burnout has been identified as a predictor of various negative
consequences for employee health and wellbeing and for organizational
success. These include physical consequences (cardiovascular diseases,
pain, and impaired immune function), psychological consequences
(depression and insomnia), and occupational consequences (absenteeism,
poor performance, and job dissatisfaction). Given the negative impacts
of burnout on employees and organizations, managers and administrators
should consider preventative measures to mitigate burnout. Job control
may be one area worth focusing mitigation measures.

Librarians may lack job control generally and when providing library
instruction specifically. There is no current data about job control
among librarians, though job control does appear to be tied to burnout.
However, little scholarly research considers librarian agency or begins
to understand the factors that contribute to agency or feelings of
agency for librarians at work. In order to improve job control and
mitigate burnout in the workplace, we first need to understand how job
control is experienced and what factors impact that experience.

Given the relationship between job control and burnout and the negative
impacts of burnout, it stands to reason that managers and administrators
should work with employees to increase their job control as a means of
mitigating burnout (Salvagioni et al., 2017). However, further research
on job control and burnout among librarians is needed.

This study seeks to address this research problem and fill the gap
identified in the literature around librarian perceptions of job control
generally and regarding instruction specifically. The study considers
the following research questions:

\begin{itemize}
\tightlist
\item
  For academic instruction librarians, how does job control impact
  burnout?
\item
  What factors contribute to job control and burnout for academic
  instruction librarians? To what extent do these factors contribute to
  job control?
\end{itemize}

This study uses the Copenhagen Burnout Inventory (CBI) and Job Control
Inventory to show how job control impacts burnout. Using the CBI,
academic instruction librarians, on average, have drastically high
work-related burnout and even higher personal burnout compared to other
jobs. However, librarians have drastically low client-related burnout,
similar to other ``caring'' or ``helping'' professions. I argue that
this difference is related to vocational awe and that person-centered
management is necessary to approach employees holistically to mitigate
personal and work-related burnout, which are statistically correlated.
Additionally, the findings point to key factors that impact job control
and burnout to help consider ways of mitigating burnout and increasing
job control.

\subsection{Literature Review}\label{literature-review}

\subsubsection{Burnout}\label{burnout}

The World Health Organization (WHO) in their International
Classification of Diseases, 11th Edition (ICD-11) describe burnout as
``a syndrome conceptualized as resulting from chronic workplace stress
that has not been successfully managed. It is characterised by three
dimensions: 1) feelings of energy depletion or exhaustion; 2) increased
mental distance from one's job, or feelings of negativism or cynicism
related to one's job; and 3) a sense of ineffectiveness and lack of
accomplishment'' (World Health Organization, 2020). For several decades,
burnout has been a preoccupation of librarians, the profession, and our
professional literature. In fact, as Wood et al. (2020) demonstrate a
scholarly literature search for burnout AND librar*, scholarly
literature on burnout in librarians has had a steady upward trend for
the past 4 decades. This has included considerable anecdotal evidence.
In fact, over 30 years ago, Fisher (1990) called for further empirical
evidence on burnout in librarians to answer her titular question ``are
librarians burning out?'' While the extent of the anecdotal evidence
should give us cause to believe librarians and a librarian's belief that
they are burnt out seems just as important as an empirical decision or
diagnosis that they are, anecdotal approaches may leave us lacking as we
attempt to understand the systemic and structural causes of burnout in
libraries and mitigate these effects. Further quantitative and
qualitative study of burnout among librarians will allow us to pinpoint
solutions for library administration to make organization and structural
changes for the benefit of library workers.

Employing the Copenhagen Burnout Inventory, Wood et al. (2020) found
that librarian perceptions of burnout are quite high in comparison to
other occupations, including nurses, hospital doctors, and social
workers. This points to a significant issue that needs to be addressed.
Considering job control as a component of agency may be one way to
measure a specific aspect of burnout and mitigate feelings of workplace
burnout. Applying the Areas of Worklife Survey (AWS) and Maslach Burnout
Inventory (MBI), Nardine (2019) found ``that lack of personal agency is
the primary contributor to a sense of burnout (p.~508 ) However, the AWS
doesn't measure agency directly, but rather what Leiter \& Maslach
(2003) refer to as Control, which''includes employee's perceived
capacity to influence decisions that affect their work, to exercise
professional autonomy, and to gain access to resources necessary to do
an effective job'' (p.~96).

\subsubsection{Job Control}\label{job-control}

Ganster (1989) defines control ``as the ability to exert some influence
over one's environment so that the environment becomes more rewarding or
less threatening.'' Job control may have the following domains or
dimensions hypothesized as ``areas from which stress at work may
arise'': work tasks, work pacing, work scheduling, physical environment,
decision making, interaction, and mobility. He also points to a
tradition of ``employee participation in decision making'' as an aspect
of job control, which previous literature on burnout has pointed to as a
solution {[}Sheesley (2001); Christian (2015); Maslach (2017); (Corrado,
2022){]}. Maslach \& Leiter (2016), identify job control, in relation to
burnout and stress, as ``the perceived capacity to influence decisions
that affect their work, to exercise professional autonomy, and to gain
access to the resources necessary to do an effective job.''

Leiter \& Maslach's (2003) conception of control in the AWS used by
(Nardine, 2019) relies on the Job Demand-Control (JDC) model (Karasek,
1979). The JDC model considers job demands (workload, pressure) and job
control (sometimes decision latitude, a worker's ability to control
their work) and proposes that a high demands-low control job will result
in greater physical and psychological stress, which suggests that
increased job control can ``buffer'' the negative impacts of increased
job demands (Van der Doef \& Maes, 1999, p. 89). While it appears
unlikely that this buffering hypothesis is true (Mansell \& Brough,
2005), existing research supports the connection between low job control
and increased burnout (Park et al., 2014; Portoghese et al., 2014; Taris
et al., 2005).

\subsubsection{Vocational Awe}\label{vocational-awe}

Fobazi Ettarh (2018) describes vocational awe as ``the set of ideas,
values, and assumptions librarians have about themselves and the
profession that result in beliefs that libraries as institutions are
inherently good and sacred, and therefore beyond critique.'' She argues
that this positioning of the library as inherently good and thus of the
workers in the library as the doers of that good work creates a
situation in which any failure of the library is a failure of the
individual ``to live up to the ideals of the profession.'' Ettarh
argues, in particular, that burnout is one of several negative impacts
caused by vocational awe. This sacredness of libraries becomes a way for
institutions to deflect criticism and avoid caring for workers, pushing
instead of individualized solutions to burnout that bely structural and
systemic issues in library organizations and libraries broadly:
``institutional response to burnout is the output of more `love and
passion,' through the vocational impulses noted earlier and a
championing of techniques like mindfulness and `whole-person'
librarianship.'' Martyrdom and self-sacrifice become features of the
profession; these are necessary features to operate libraries that are
understaffed and under-resourced---doing more with less: ``Awe is easily
weaponized against the worker, allowing anyone to deploy a vocational
purity test in which the worker can be accuse of not being devout or
passionate enough to serve without complaint.'' Workers may even
weaponize awe against themselves as a form of self-regulation to meet
the unrealistic ideals of the profession.

\subsection{Materials and Methods}\label{materials-and-methods}

A web survey was administered to measure job control and burnout among
academic librarians with instruction responsibilities. To measure job
control, the survey used the job control measure designed and validated
by Dwyer \& Ganster (1991), which includes 22 questions. To measure
burnout, the survey used the Copenhagen Burnout inventory described by
Kristensen, et al. (n.d.), which includes 19 questions.

\subsubsection{Sample and Recruitment}\label{sample-and-recruitment}

The target population for the study was academic librarians with some
instruction responsibilities. The survey was distributed using
professional distribution lists provided by the ALA Connect platform
that operates as a forum and email distribution system. The recruitment
email was sent three times (29 August 2022, 13 September 2022, and 28
September 2022) with concurrent messages via the social media platform
Twitter. To participate, individuals needed to be currently employed in
an academic library and have at least some teaching responsibilities.
Calculating the reach of these methods and who within that reach meets
the participation requirement is difficult; however, the ALA Connect
distribution was sent to three lists: ACRL Members, which includes
approximately 7,200 members; ACRL Instruction Section, which includes
4,800 members; and Information Literacy Instruction in Academic
Libraries, which includes 292 members. Given the size of the field and
the connections between these groups, there is certainly overlap among
the population across these three lists. In the end, 307 survey
responses were collected, of which, 245 included complete results, which
were used for data analysis. Demographic characteristics of the sample
are included in Table 1. Participants could select more than one
response for sexuality and race and ethnicity.

Table 1. Demographic characteristics of the sample.

\begin{longtable}[]{@{}
  >{\raggedright\arraybackslash}p{(\columnwidth - 4\tabcolsep) * \real{0.8068}}
  >{\raggedright\arraybackslash}p{(\columnwidth - 4\tabcolsep) * \real{0.0568}}
  >{\raggedright\arraybackslash}p{(\columnwidth - 4\tabcolsep) * \real{0.1364}}@{}}
\toprule\noalign{}
\begin{minipage}[b]{\linewidth}\raggedright
Characteristic
\end{minipage} & \begin{minipage}[b]{\linewidth}\raggedright
No.
\end{minipage} & \begin{minipage}[b]{\linewidth}\raggedright
Percentage
\end{minipage} \\
\midrule\noalign{}
\endhead
\bottomrule\noalign{}
\endlastfoot
\textbf{Gender} & & \\
Agender & 2 & 0.82 \\
Genderqueer or gender fluid & 3 & 1.22 \\
Man & 22 & 8.98 \\
Nonbinary & 3 & 1.22 \\
Prefer not to say & 5 & 2.04 \\
Unsure & 4 & 1.63 \\
Woman & 206 & 84.08 \\
& & \\
\textbf{Gender Modality} & & \\
Cisgender & 228 & 93.06 \\
Prefer not to disclose & 12 & 4.90 \\
Transgender & 3 & 1.22 \\
Unsure & 1 & 0.41 \\
Missing & 1 & 0.41 \\
& & \\
\textbf{Sexuality} & & \\
Asexual & 15 & 6.12 \\
Bisexual & 39 & 15.92 \\
Gay & 5 & 2.04 \\
Lesbian & 7 & 2.86 \\
Pansexual & 7 & 2.86 \\
Queer & 18 & 7.35 \\
Straight & 158 & 64.49 \\
Prefer not to disclose & 15 & 6.12 \\
& & \\
\textbf{Disability} & & \\
Abled & 188 & 76.73 \\
Disabled & 44 & 17.96 \\
Prefer not to disclose & 12 & 4.90 \\
missing & 1 & 0.41 \\
& & \\
\textbf{Race \& Ethnicity} & & \\
African & 1 & 0.41 \\
African American/Black & 6 & 2.45 \\
East Asian & 1 & 0.41 \\
Hispanic or Latinx/Latine & 12 & 4.90 \\
Indigenous American, Native American, First Nations, or Alaska Native &
2 & 0.82 \\
Middle Eastern or North African & 4 & 1.63 \\
Southeast Asian & 1 & 0.41 \\
White & 222 & 90.61 \\
Prefer not to disclose & 10 & 4.08 \\
\end{longtable}

\subsubsection{Measures}\label{measures}

The web survey (developed using LibWizard) included demographic
questions, questions about the characteristics of the participant's
job/employment, and two validated scales related to work.

The first of these validated scales was on job control, which was
developed and validated by Ganster (1989) and (Dwyer \& Ganster, 1991).
The inventory includes 22 questions to measure job control across
various dimensions. In Ganster (1989), the first 21 questions are used
to measure job control, and question 22 is used as a control; however,
in (Dwyer \& Ganster, 1991), the authors use all 22 questions to
calculate the job control score. Participants were asked all 22
questions; however, in this study, the first 21 questions are used to
calculate the job control score. Participants were asked to complete
this job control inventory as it applies to their job generally, and
then asked again to complete the same job control inventory but thinking
specifically about their instruction responsibilities or the
instructional aspects of their roles. However, the questions were
exactly the same both times. Despite the note about this in the survey,
this may have resulted in fewer complete responses. Scoring for the job
control inventory uses a Likert scale with values 1 through five
attributed (Very little = 1; Little = 2; A moderate amount = 3; Much =
4; and Very much =5). The job control score is the average of these for
the participant across the 21 items in the inventory.

The Chronbach's alpha for the 21 item job control scale was 0.89 (n=245)
when used for job control in general and 0.894 (n=245) when used for job
control specifically related to instruction. Adding the twenty-second
item increases the Chronbach's alphas to 0.899 and 0.902 respectively;
however, the internal consistency is still good with the 21-item scale,
and the twenty-second item was meant as an overall control for
perception. This is also similar to Ganster's original 1989 report on
the scale, which had an alpha of 0.87 (n = 191), and Dywer \& Ganster
(1991), which also had an alpha of 0.87 (n = 90).

The second of these validated scales was on burnout, using the
Copenhagen Burnout Inventory (CBI), which includes three subscales:
personal burnout (6 items), work-related burnout (7 items), and
client-related burnout (6 items). For the purposes of this study, the
word client in the client-related burnout subscale was changed to
``patrons,'' as it was believed that this terminology was better aligned
with how librarians consider users. This is aligned with general usage
of the CBI: ``\,`Clients' is a broad concept covering terms such as
patients, inmates, children, students, residents, etc. When the CBI is
used in practice, the term appropriate for the specific group of
respondents is used'' (Kristensen et al., n.d.).

Kristensen et al. (n.d.) define these three dimensions measured by the
subscales as follows:

\begin{itemize}
\tightlist
\item
  \textbf{Personal burnout:} ``the degree of physical and psychological
  fatigue and exhaustion experienced by the person''
\item
  \textbf{Work-related burnout:} ``the degree of physical and
  psychological fatigue and exhaustion that is perceived by the person
  as related to his/her work''
\item
  \textbf{Client-related burnout:} ``the degree of physical and
  psychological fatigue and exhaustion that is perceived by the person
  as related to his/her work with clients''
\end{itemize}

Thus, personal burnout is not necessarily related to an individual's
personal life but rather to a more general or generic assessment of
burnout.

The CBI uses two different Likert scales that are given values ranging
from 0 to 100, and one question in the work-related burnout inventory is
inversely scored. The total work-related burnout score (TWRBS), total
personal burnout score (TPBS), and total client-related burnout score
(TCRBS) are the average within the given subscale for the participant.

The Chronbach's alpha for the personal burnout subscale, work-related
burnout subscale, and client-related burnout subscale from the
Copenhagen Burnout Inventory were 0.875, 0.889, and 0.887 respectively,
which is similar to Kristensen et al.~(2005), which reported a range
from 0.85 to 0.87 (n = 1,910), and Wood et al.~(2020) with a Chronbach's
alpha of 0.798 (n = 1,808) for the work-related burnout subscale. The
results demonstrate that these subscales have a good measure of
reliability as well.

Finally, the survey included questions about demographics and questions
about the characteristics of the participants employment. The following
demographic questions were included:

\begin{enumerate}
\def\labelenumi{\arabic{enumi}.}
\tightlist
\item
  What is your gender?
\item
  What is your gender modality? ``Gender modality refers to how a
  person's gender identity stands in relation to their gender assigned
  at birth'' (Ashley, 2022)
\item
  What is your sexuality? Select all that apply.
\item
  Are you disabled?
\item
  Which of the identities described above have you disclosed at work or
  would you consider to be ``out'' at work? Select all that apply.
\item
  Please describe your race/ethnicity. Select all that apply.
\end{enumerate}

A summary of this demographic information within the sample is available
in Table 1 above.

The following questions about the characteristics of the participants
employment were included because it was hypothesized that they would
have an effect on job control or burnout:

\begin{enumerate}
\def\labelenumi{\arabic{enumi}.}
\tightlist
\item
  How long (in years) have you worked at your current institution?
\item
  How long (in years) have you been in a librarian position after
  receiving your degree (in library science or equivalent)?
\item
  How long (in years) have you worked in libraries in any capacity?
\item
  Which of the following best describes the institution where you work?
\item
  Which of the following best describes your current position?
\item
  What is your annual salary or income (before taxes, etc.) in US
  Dollars?
\item
  Which of the following best describes your employment status at your
  current institution?
\item
  Are librarians at your institution eligible for tenure or an
  equivalent status?
\item
  Have you obtained tenure or its equivalent at your institution? (This
  was only revealed if the participant answered yes to the previous
  question.)
\item
  Are you represented by a union?
\item
  Have you received formal training in library school or on the job
  specifically intended to prepare you to teach?
\item
  Do you believe this training adequately prepared you for teaching?
  (This was only revealed if the participant answered yes to the
  previous question.)
\item
  Which of the following best describes your teaching workload?
\end{enumerate}

A summary of these characteristics within the sample is included in
Table 2 below.

Table 2. Summary of job characteristics for participants in the sample.

\begin{longtable}[]{@{}
  >{\raggedright\arraybackslash}p{(\columnwidth - 4\tabcolsep) * \real{0.7927}}
  >{\raggedright\arraybackslash}p{(\columnwidth - 4\tabcolsep) * \real{0.0610}}
  >{\raggedright\arraybackslash}p{(\columnwidth - 4\tabcolsep) * \real{0.1463}}@{}}
\toprule\noalign{}
\begin{minipage}[b]{\linewidth}\raggedright
Characteristic
\end{minipage} & \begin{minipage}[b]{\linewidth}\raggedright
No.
\end{minipage} & \begin{minipage}[b]{\linewidth}\raggedright
Percentage
\end{minipage} \\
\midrule\noalign{}
\endhead
\bottomrule\noalign{}
\endlastfoot
\textbf{Length of time at current institution (in years)} & & \\
Less than 1 & 31 & 12.65 \\
1 to 5 & 100 & 40.82 \\
6 to 10 & 55 & 22.45 \\
11 to 15 & 23 & 9.39 \\
16 or more & 36 & 14.69 \\
& & \\
\textbf{Length of time since obtaining their degree (in years)} & & \\
Less than 1 & 6 & 2.45 \\
1 to 5 & 62 & 25.31 \\
6 to 10 & 63 & 25.71 \\
11 to 15 & 45 & 18.37 \\
16 or more & 67 & 27.35 \\
missing & 2 & 0.82 \\
& & \\
\textbf{Length of time working in libraries (in years)} & & \\
1 to 5 & 31 & 12.65 \\
6 to 10 & 61 & 24.90 \\
11 to 15 & 57 & 23.27 \\
16 or more & 93 & 37.96 \\
missing & 3 & 1.22 \\
& & \\
\textbf{Type of institution} & & \\
Associate's college & 28 & 11.43 \\
Baccalaureate college & 30 & 12.24 \\
Doctoral university & 130 & 53.06 \\
Law school & 3 & 1.22 \\
Master's college or university & 54 & 22.04 \\
& & \\
\textbf{Public or private} & & \\
Private & 91 & 37.14 \\
Public & 154 & 62.86 \\
& & \\
\textbf{For-profit or non-profit} & & \\
For-profit & 1 & 0.41 \\
Non-profit & 244 & 99.59 \\
& & \\
\textbf{Permanent or temporary position} & & \\
Permanent & 240 & 97.96 \\
Probationary & 1 & 0.41 \\
Temporary & 4 & 1.63 \\
& & \\
\textbf{Full-time or part-time} & & \\
Full-time & 242 & 98.78 \\
Part-time & 3 & 1.22 \\
& & \\
\textbf{Income} & & \\
\$20,000 to \$34,999 & 1 & 0.41 \\
\$35,000 to \$49,999 & 16 & 6.53 \\
\$50,000 to \$74,999 & 129 & 52.65 \\
\$75,000 to \$99,999 & 77 & 31.43 \\
\$100,000 or greater & 16 & 6.53 \\
Prefer not to disclose & 6 & 2.45 \\
& & \\
\textbf{Faculty status} & & \\
Academic staff & 58 & 23.67 \\
Faculty & 153 & 62.45 \\
Staff & 34 & 13.88 \\
& & \\
\textbf{For faculty, tenure status (n=153)} & & \\
Non-tenure-track & 60 & 39.22 \\
Tenure-track & 92 & 60.13 \\
Tenured & 1 & 0.65 \\
& & \\
\textbf{Tenure for librarians at institution} & & \\
No & 120 & 48.98 \\
Yes, similar status & 34 & 13.88 \\
Yes, tenure & 86 & 35.1 \\
Other & 5 & 2.04 \\
& & \\
\textbf{Tenure status for individual participant (n=120)} & & \\
No & 63 & 52.5 \\
Yes, I am tenured & 36 & 30 \\
Yes, I have attained an equivalent status & 19 & 15.83 \\
Other & 2 & 1.67 \\
& & \\
\textbf{Union status} & & \\
In the process of unionizing & 3 & 1.22 \\
No & 169 & 68.98 \\
Unsure & 5 & 2.04 \\
Yes & 67 & 27.35 \\
Other & 1 & 0.41 \\
& & \\
\textbf{Training for library instruction} & & \\
No & 58 & 23.67 \\
Yes, in library school and on the job & 88 & 35.92 \\
Yes, only in library school & 40 & 16.33 \\
Yes, only on the job & 45 & 18.37 \\
Other & 14 & 5.71 \\
& & \\
\textbf{Perception of effectiveness of training preparation (n=172)} &
& \\
Highly & 41 & 23.84 \\
Not at all & 20 & 11.63 \\
Somewhat & 111 & 64.53 \\
& & \\
\textbf{Perception of teaching workload} & & \\
Far too excessive & 10 & 4.08 \\
Slightly excessive & 68 & 27.76 \\
Just right & 90 & 36.73 \\
Slightly light & 60 & 24.49 \\
Far too light & 17 & 6.94 \\
\end{longtable}

It is unclear why only one participant identified their position as
tenure faculty, but 86 participants said they had tenure. The wording of
the questions was likely confusing.

\subsubsection{Statistical Analyses}\label{statistical-analyses}

Analyses were conducted using the R Statistical language {[}version
4.2.1; R Core Team (2022){]} on macOS Monterey 12.5.1, using the
packages easystats {[}version 0.5.2; Lüdecke et al. (2022){]}, ltm
{[}version 1.2.0; Rizopoulos (2007){]}, MASS {[}version 7.3.58.1;
Venables \& Ripley (2002){]}, plyr {[}version 1.8.8; Wickham (2011){]},
ggplot2 {[}version 3.4.0; Wickham (2022a){]}, stringr {[}version 1.4.1;
Wickham (2022b){]}, dplyr {[}version 1.0.10; Wickham et al. (2022){]},
and tidyr {[}version 1.2.1; Wickham \& Girlich (2022){]}.

\subsubsection{Ethical Considerations}\label{ethical-considerations}

Human research ethics approval was obtained from the Institutional
Review Board at the University of California, Los Angeles
(IRB\#22-001337), which certified the study as exempt. Consent was
implied by participants clicking a button labelled ``I agree to
participate'' at the start of the survey after reading an information
sheet concerning the study. No survey responses were required, so
participants could simply skip any question; however, many questions
also gave an option for ``prefer not to disclose'' as well.

\phantomsection\label{refs}
\begin{CSLReferences}{1}{0}
\bibitem[\citeproctext]{ref-ashleyTransMyGender2022}
Ashley, F. (2022). '{Trans}' is my gender modality: {A} modest
terminological proposal. In L. Erickson-Schroth (Ed.), \emph{Trans
bodies, trans selves: {A} resource by and for transgender communities}
(Second Edition). Oxford University Press.

\bibitem[\citeproctext]{ref-christianPassionDeficitOccupational2015}
Christian, L. (2015). A {Passion Deficit}: {Occupational Burnout} and
the {New Librarian}: {A Recommendation Report}. \emph{The Southeastern
Librarian}, \emph{62}(4).

\bibitem[\citeproctext]{ref-corradoLowMoraleBurnout2022}
Corrado, E. M. (2022). Low {Morale} and {Burnout} in {Libraries}.
\emph{Technical Services Quarterly}, \emph{39}(1), 37--48.
\url{https://doi.org/10.1080/07317131.2021.2011149}

\bibitem[\citeproctext]{ref-dwyerEffectsJobDemands1991}
Dwyer, D. J., \& Ganster, D. C. (1991). The effects of job demands and
control on employee attendance and satisfaction. \emph{Journal of
Organizational Behavior}, \emph{12}(7), 595--608.
\url{https://doi.org/10.1002/job.4030120704}

\bibitem[\citeproctext]{ref-karasekJobDemandsJob1979}
Karasek, R. A. (1979). Job {Demands}, {Job Decision Latitude}, and
{Mental Strain}: {Implications} for {Job Redesign}. \emph{Administrative
Science Quarterly}, \emph{24}(2), 285--308.
\url{https://doi.org/10.2307/2392498}

\bibitem[\citeproctext]{ref-kristensenCopenhagenBurnoutInventory}
Kristensen, T. S., Borritz, M., Villadsen, E., \& Christensen, K. B.
(n.d.). \emph{Copenhagen {Burnout Inventory}}.
\url{https://doi.org/10.1037/t62096-000}

\bibitem[\citeproctext]{ref-leiterAreasWorklifeStructured2003}
Leiter, M. P., \& Maslach, C. (2003). Areas of {Worklife}: {A Structured
Approach} to {Organizational Predictors} of {Job Burnout}. In P. L.
Perrewe \& D. C. Ganster (Eds.), \emph{Emotional and {Physiological
Processes} and {Positive Intervention Strategies}} (Vol. 3, pp.
91--134). Emerald Group Publishing Limited.
\url{https://doi.org/10.1016/S1479-3555(03)03003-8}

\bibitem[\citeproctext]{ref-ludeckeFrameworkEasyStatistical2022}
Lüdecke, D., Makowski, D., Ben-Shachar, M. S., Patil, I., \& Wiernik, B.
M. (2022). \emph{Framework for {Easy Statistical Modeling},
{Visualization}, and {Reporting}}.

\bibitem[\citeproctext]{ref-mansellComprehensiveTestJob2005}
Mansell, A., \& Brough, P. (2005). A comprehensive test of the job
demands-control interaction: {Comparing} two measures of job
characteristics. \emph{Australian Journal of Psychology}, \emph{57}(2),
103--114. \url{https://doi.org/10.1080/10519990500048579}

\bibitem[\citeproctext]{ref-maslachFindingSolutionsProblem2017}
Maslach, C. (2017). Finding solutions to the problem of burnout.
\emph{Consulting Psychology Journal: Practice and Research}, \emph{69},
143--152. \url{https://doi.org/10.1037/cpb0000090}

\bibitem[\citeproctext]{ref-maslachChapter43Burnout2016}
Maslach, C., \& Leiter, M. P. (2016). Chapter 43 - {Burnout}. In G. Fink
(Ed.), \emph{Stress: {Concepts}, {Cognition}, {Emotion}, and {Behavior}}
(pp. 351--357). Academic Press.
\url{https://doi.org/10.1016/B978-0-12-800951-2.00044-3}

\bibitem[\citeproctext]{ref-nardineStateAcademicLiaison2019}
Nardine, J. (2019). The state of academic liaison librarian burnout in
{ARL} libraries in the united states. \emph{College \& Research
Libraries}, \emph{80}(4).

\bibitem[\citeproctext]{ref-parkJobControlBurnout2014}
Park, H. I., Jacob, A. C., Wagner, S. H., \& Baiden, M. (2014). Job
control and burnout: {A Meta}-analytic test of the {Conservation} of
{Resources} model. \emph{Applied Psychology: An International Review},
\emph{63}(4), 607--642. \url{https://doi.org/10.1111/apps.12008}

\bibitem[\citeproctext]{ref-portogheseBurnoutWorkloadHealth2014}
Portoghese, I., Galletta, M., Coppola, R. C., Finco, G., \& Campagna, M.
(2014). Burnout and {Workload Among Health Care Workers}: {The
Moderating Role} of {Job Control}. \emph{Safety and Health at Work},
\emph{5}(3), 152--157. \url{https://doi.org/10.1016/j.shaw.2014.05.004}

\bibitem[\citeproctext]{ref-rcoreteamProjectStatisticalComputing2022}
R Core Team. (2022). \emph{R: {The R Project} for {Statistical
Computing}}. R Foundation for Statistical Computing.

\bibitem[\citeproctext]{ref-rizopoulosLtmPackageLatent2007}
Rizopoulos, D. (2007). Ltm: {An R Package} for {Latent Variable
Modeling} and {Item Response Analysis}. \emph{Journal of Statistical
Software}, \emph{17}, 1--25. \url{https://doi.org/10.18637/jss.v017.i05}

\bibitem[\citeproctext]{ref-salvagioniPhysicalPsychologicalOccupational2017}
Salvagioni, D. A. J., Melanda, F. N., Mesas, A. E., González, A. D.,
Gabani, F. L., \& Andrade, S. M. de. (2017). Physical, psychological and
occupational consequences of job burnout: {A} systematic review of
prospective studies. \emph{PLOS ONE}, \emph{12}(10), e0185781.
\url{https://doi.org/10.1371/journal.pone.0185781}

\bibitem[\citeproctext]{ref-sheesleyBurnoutAcademicTeaching2001}
Sheesley, D. F. (2001). Burnout and the academic teaching librarian:
{An} examination of the problem and suggested solutions. \emph{The
Journal of Academic Librarianship}, \emph{27}(6), 447--451.
\url{https://doi.org/10.1016/S0099-1333(01)00264-6}

\bibitem[\citeproctext]{ref-tarisJobControlBurnout2005}
Taris, T. W., Bakker, A. B., Schaufeli, W. B., Stoffelsen, J., \& Van
Dierendonck, D. (2005). Job {Control} and {Burnout} across
{Occupations}. \emph{Psychological Reports}, \emph{97}(3), 955--961.
\url{https://doi.org/10.2466/pr0.97.3.955-961}

\bibitem[\citeproctext]{ref-vanderdoefJobDemandControlSupport1999}
Van der Doef, M., \& Maes, S. (1999). The {Job Demand-Control}
(-{Support}) {Model} and psychological well-being: {A} review of 20
years of empirical research. \emph{Work \& Stress}, \emph{13}(2),
87--114. \url{https://doi.org/10.1080/026783799296084}

\bibitem[\citeproctext]{ref-venablesModernAppliedStatistics2002}
Venables, W. N., \& Ripley, B. D. (2002). \emph{Modern {Applied
Statistics} with {S}, 4th ed} (4th ed.). Springer.

\bibitem[\citeproctext]{ref-wickhamSplitApplyCombineStrategyData2011}
Wickham, H. (2011). The {Split-Apply-Combine Strategy} for {Data
Analysis}. \emph{Journal of Statistical Software}, \emph{40}, 1--29.
\url{https://doi.org/10.18637/jss.v040.i01}

\bibitem[\citeproctext]{ref-wickhamGgplot2ElegantGraphics2022}
Wickham, H. (2022a). \emph{Ggplot2: {Elegant} graphics for data
analysis}. Springer-Verlag New York.

\bibitem[\citeproctext]{ref-wickhamStringrSimpleConsistent2022}
Wickham, H. (2022b). \emph{Stringr: {Simple}, {Consistent Wrappers} for
{Common String Operations}}.

\bibitem[\citeproctext]{ref-wickhamDplyrGrammarData2022}
Wickham, H., François, R., Henry, L., \& Müller, K. (2022). \emph{Dplyr:
{A Grammar} of {Data Manipulation}}.

\bibitem[\citeproctext]{ref-wickhamTidyrTidyMessy2022}
Wickham, H., \& Girlich, M. (2022). \emph{Tidyr: {Tidy Messy Data}}.

\bibitem[\citeproctext]{ref-woodAcademicLibrarianBurnout2020}
Wood, B. A., Guimaraes, A. B., Holm, C. E., Hayes, S. W., \& Brooks, K.
R. (2020). Academic librarian burnout: {A} survey using the {Copenhagen
Burnout Inventory} ({CBI}). \emph{Journal of Library Administration},
\emph{60}(5), 512--531.
\url{https://doi.org/10.1080/01930826.2020.1729622}

\end{CSLReferences}



\end{document}
